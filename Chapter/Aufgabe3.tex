\chapter{Aufgabe 3}
\section{Datentypen Allgemein}
In der Programmierung sind Datentypen ein grundlegendes Konzept.
Sie geben die "Art" einer Variablen an und auch welche Operationen mit diesen durchgeführt werden können und welche zu Fehlern führen.
Mit einem character ist es beispielsweise nicht möglich mathematische Operationen durchzuführen.
Sie geben auch an wie viel Speicher für diese reserviert werden muss\cite{datentypen:2022} 

\section{Datentypen in der Programmiersprache C}
Die Programmiersprache C ist eine \textbf{stark typisierte Sprache} und sind dort besonders wichtig.
Jeder Ausdruck und jeder Wert ist einem bestimmten Datentypen zugeordnet.
In C gibt es fundamentale Typen wie Char, Float und Integer.
Diese können noch genauer beschrieben werden, indem diese mit signed, unsigned, long oder short beschrieben werden.
Es gibt in C auch Wertlose Datentypen namens Void.
Weiterhin gibt es die abgeleiteten Datentypen wie Arrays, Pointer und Function.
Ebenfalls eine wichtige Rolle spielen Nutzerdefinierte Datentypen wie Struct, Union und Enum\cite{boekelmann:2023}.

\section{Nutzerdefinierte Datentypen}
Nutzerdefinierte Typen eignen sich dafür eine Menge an unterschiedlichen Informationen in einem Objekt zu verbinden.
Dabei können die \textit{Member} unterschiedliche Datentypen innerhalb eines Strutcs besitzen\cite{boekelmann:2023}.
Somit lassen sich Daten die zusammengehören in einem Objekt verbinden.
Ein weiterer Vorteil ist das die Lesbarkeit des Codes erhöht wird.
An Stelle vieler einzelner initialisierter Variablen erhält man eine gekapselte Struktur in der die zusammengehörenden Daten gelistet sind.
Die Übersicht im Code wird erhöht und die Zusammengehörigkeit deutlich gemacht.
Ein weiterer Vorteil ist die Übergabe an Funktionen.
Übergibt man eine Struktur an eine Funktion können mehrere Daten an diese übergeben werden.
Dies erhöht ebenfalls die Lesbarkeit, da diese Daten sonst alle einzeln übergeben werden müssten.
Dies kann ebenfalls denn Code effizienter machen.

\section{typedef}
Das Keyword \textit{typedef} kann genutzt werden, um einen Alias für einen existierenden Datentypen zu erstellen.
Dies hilft nicht nur dabei den Code lesbarer zu gestalten sondern kann auch im Programmierverlauf die Schreibarbeit erleichtern\cite{boekelmann:2023}.

\section{Repo zur Aufgabe 3}
In dem Repo auf Gitlab befindet sich die Bearbeitung des Programmierteils der Aufgabe 3. 
In dieser Aufgabe soll ein struct erstellt werden, das die physikalisch messbaren Merkmale einer Oszilloskop Kurve darstellt.\par
\href{https://gitlab.thga.de/daniel.krueger/pruefung_sose_2023_aufgabe_3_struct}{\textbf{LINK}} 
