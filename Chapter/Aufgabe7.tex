\chapter{Aufgabe 7}
\section{POSIX und Streams}
\textit{POSIX} ist die kurzform des \textit{Portable Operating System Interface}.
Unter diesem Namen ist eine Reihe von IEEE Standards für Unix spezifiziert worden.
Dabei handelt es sich um eine standardisierte Programmierschnittstelle, die die Schnittstelle zwischen Anwendungen und Betriebssystem darstellen soll\cite{posix:2023}.
Bei diesen liegt der Hauptfokus auf File I/O und Streams.\par
Im POSIX wird unterschieden zwischen Datenquellen und -senken.
Dabei kann man die die Datenquellen als Sender und die Datensenken als Empfänger sehen.
Diese beiden werden über die sogenannten Streams, also Datenströme, verbunden.
Datenströme oder Streams sind kontinuierliche Byteströme.
Geräte sind in der Lage diese Byteströme zu lesen oder zu schreiben.
Wenn nun zwei Geräte miteinander kommunizieren oder Daten austauschen wollen, kann eines dieser Geräte in den Stream schreiben (Datenquelle) und das andere diesen lesen (Datensenke).
Mit Streams kann ebenfalls mit Dateien kommuniziert werden.\par
In POSIX wird alles als Datei angesehen. 
Jede Datensenke hat einen \textit{Filename}.\par
In hosted Implementierungen greifen Programme nicht auf Dateien zu.
Sie kommunizieren über Streams nach außen.
Die Programme bitten das Betriebssystem einen Stream des Inhaltes zur Verfügung zu stellen.
Diese Aufgabe übernimmt der File-Descriptor.
Dieser enthält die Informationen über die Verknüpfung eines Streams.
Dieser wird vom Betriebssystem erstellt und repräsentiert den geöffneten Stream.
Die File-Descriptor werden durchnummeriert mit Integer Werten damit diese den geöffneten Streams im Prozess zugeordnet werden können\cite{filedescriptors:2011}.\par
Bei freestanding Implementierungen ist dies komplizierter, da es kein Betriebssystem gibt das diesen Stream zur Verfügung stellt.
Dort muss die Codierung Byte für Byte kontrolliert werden.\par 
Das Betriebssystem öffnet auch Standardmäßig einige Streams.
Diese sind zum Beispiel mit dem Terminal verbunden.
Der stdin Stream liest die vom Benutzer ins Terminal geschriebenen Daten und gibt diese ans Betriebssystem weiter.
Der stdout Stream bekommt hingegen Daten vom Betriebssystem und übergibt diese ans Terminal\cite{boekelmann:2023}.


\section{Repo} 
In diesem Repo auf Gtilab befindet sich die Bearbeitung des Programmierteils der Aufgabe 7.
In diesem C-Programm soll ein Beispiel für das arbeiten mit Streams dargestellt werden.
Dabei wird aus einem Stream gelesen und etwas in einen Stream geschrieben.\par
\href{https://gitlab.thga.de/daniel.krueger/pruefung_sose_2023_aufgabe_7_streams}{\textbf{LINK}}
