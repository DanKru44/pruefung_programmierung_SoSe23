\section{Textsatzbeschreibungssprachen}
Mit dem Begriff \textit{Satz} wird ein technisches Verfahren beschrieben mit dem aus einer Vorlage ein drucktaugliche Form hergestellt wird. Diese können händische, maschinelle oder auch computergestütze Verfahren sein. Textsatzbeschreibungssprachen gehören zu letzterem und genauer zu Art \textit{Struktursatz}. Dabei liegt die setzterische Leistung in der Gestaltung einer Vorlage, die dann über ein technisches Regelwerk in ein Satzsystem einfließen und automatisiert \textit{layoutet} wird. Ziel des Textsatz ist es ein Dokument \textbf{ästhetisch ansprechend} und \textbf{leicht leserlich} zu formatiern. Dafür werden zum Beispiel Schriftarten und -größe, Zeilenabstände und das Seitenlayout festgelegt\nocite{satzwiki:2023}. Diese Prinzip wird auch mit \textit{What you see is what you ask for (WYSIWYAF)} umschrieben.
TeX und auch LaTeX sind dafür häufig verwendete Texsatzbeschreibungssprachen. TeX wurde 1978 von Donald E. Knuth entwickelt und LaTeX ist ein von Leslie Lamport für TeX entwickeltes Softwarepacket das \textbf{die Benutzung von TeX vereinfacht}\nocite{tex:2023} \nocite{latex:2023}. Für Unix wurde 1990 das Textsatzsystem Groff von James Clark veröffentlicht. Dort wird es genutzt um \textit{manpages}(Bedienungsanleitungen) anzuzeigen\nocite{groff:2022}.

\section{Vergleich Textsatzbeschreibungssprachen und Programmiersprachen}


\section{Unterschiede zu integrierten Wortprozessoren}
Integrierte Wordprozessoren arbeiten nach dem Prinzip \textit{What you see is what you get (WYSIWYG)}. Dies bedeutet das die Darstellung auf dem Bildschirm in \textbf{Echtzeitdarstellung} stattfindet. So wie es auf dem Bildschirm angezeigt wird, erfolgt auch die Ausgabe über ein anderes Gerät. Genutzt wird dies z.b. bei \textit{HTML-Editoren} um um Personen mit \textbf{wenig HTML-Kentnissen} das editieren von Webseiten zu ermöglichen. Aber auch Programme wie Word nutzen dieses Prinzip um das editieren von Texten ohne große Einarbeitungszeit einfach und schnell zu ermöglichen.
Textsatzbeschreibungssprachen setzen hingegen auf das bereits erwähnte WYSIWYAF-Prinzip. Dabei werden innerhalb des Textes formatierungseinstellungen direkt ausgezeichnet. Das Latex-System verarbeitet verarbeitet den Quellcode entsprechend den getroffenen Einstellungen. Dies bedeutet das die Ausgabe sich \textbf{unterscheidet} von dem geschriebenen Text. Dies erfordert allerdings eine \textbf{längere Einarbeitungszeit} als das WYSIWYG-Prinzip.
Dies bringt aber auch Vorteile mit sich. Textsatzbeschreibungssprachen sind häufig rechnerunabhängig. Im Beispiel für LaTeX heißt dies, es gibt für die meisten Betriebssysteme lauffähige LaTeX Distributionen mit denen gearbeitet werden kann. Auch eignen sich Textsatzbeschreibungssprachen für das erstellen von größeren Dokumentationen. LaTeX ist darauf ausgelegt gut gestaltete Dokumente mit einheitlicher Formatierung zu gestalten. Die Wiederverwendbarkeit von LaTeX-Vorlagen ist im Punkt Arbeitsgeschwindigkeit sehr gut umgesetzt. Diese Vorlagen lass sich auch nachträglich auf Dokumente anwenden.\nocite{schweitze:2013} \nocite{hommel:2010}
