\section{Textsatzbeschreibungssprachen}
Mit dem Begriff \textit{Satz} wird ein technisches Verfahren beschrieben mit dem aus einer Vorlage ein drucktaugliche Form hergestellt wird. Diese können händische, maschinelle oder auch computergestütze Verfahren sein. Textsatzbeschreibungssprachen gehören zu letzterem und genauer zu Art \textit{Struktursatz}. Dabei liegt die setzterische Leistung in der Gestaltung einer Vorlage, die dann über ein technisches Regelwerk in ein Satzsystem einfließen und automatisiert \textit{layoutet} wird. Ziel des Textsatz ist es ein Dokument \textbf{ästhetisch ansprechend} und \textbf{leicht leserlich} zu formatiern. Dafür werden zum Beispiel Schriftarten und -größe, Zeilenabstände und das Seitenlayout festgelegt.
TeX und auch LaTeX sind dafür häufig verwendete Texsatzbeschreibungssprachen. TeX wurde 1978 von Donald E. Knuth entwickelt und LaTeX ist ein von Leslie Lamport für TeX entwickeltes Softwarepacket das \textbf{die Benutzung von TeX vereinfacht}. 

\section{Unterschiede zu integrierten Wortprozessoren}
\citep{hommel:2010}
