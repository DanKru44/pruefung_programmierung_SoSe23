\section{Das heirloom-project}
Das heirloom-project wurde 2007 von Gunnar Ritter erstellt und enthält eine Sammlung von traditionellen Unix \textit{Dienstprogrammen}. Die Idee dahinter ist nicht alte Programme zu Verfügung zustellen, sondern nur bestimme Aspekte wie den Stil, Algorithmen oder das \textit{Interface} beizubehalten und für diese einen \textbf{modernisierten \textit{Framework}} zu bieten.

\section{Das Programm cat}
Der Name des Programms cat leitet sich von dem Wort concatenate ab. Dies bedeutet verketten oder verknüpfen. Daraus lässt sich auch schon schließen wo für dieses Programm gedacht ist. Es dient dazu Dateien zusammenzufügen, also dem verketten von Dateien. Häufig wird es allerdings genutzt um sich den Inhalt von Dateien auf dem Terminal auszugeben. Es wird damit eine Verkettung von Datei und Bildschirm erzielt. Dies eignet sich allerdings nur für kleineren Dateien, da sonst das Terminal schnell unübersichtlich wird. Ein guter Verwendungszweck für dieses Programm ist die Verwendung mit zusätzlichen Operatoren, wie dem \textit{Pipe-Operator} um den Inhalt in andere Dateien umzuleiten.

\subsection{Kompilation und Beobachtungen}

