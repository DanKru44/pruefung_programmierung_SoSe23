\chapter{Aufgabe 10}
\section{Das X11-Protokoll}
Das \textit{Grafiksubsystem} für \textit{UNIX-Systeme} nennt sich X11 oder \textit{X Window System}.
Die X.Org ist dabei für die Entwicklung der X Window System Standards und Spezifikationen verantwortlich.
Somit ist das X11-Protokoll standardisiert und ein X11-Server unter Windows kann auch von X11-Clients unter Linux genutzt werden.
X11 besitzt eine Client/Server-Architektur.\par
Der X11-Server stellt dabei Grafikfunktionen über seinen Serverdienst zur Verfügung.
Der X11-Server läuft lokal auf dem Rechner und verwaltet die Grafik-Hardware.
Die Dienstleistungen können zum Beispiel die Ausgabe von Textzeichen im Grafikmodus sein.
Ebenfalls kümmert sich der X-Server um die Eingabe über Peripherie wie Maus oder Tastatur.\par
Der X11-Client sind Anwendungen.
Diese nutzen die Dienstleistungen des X11-Server zur graphischen Ausgabe. \par
Somit reagiert der X11-Server auf eine Anfrage des X11-Client zur Nutzung einer seiner Dienstleistungen und stellt diese zur Verfügung.
Der X11-Client nutzt dann diese Dienstleistungen. \par
Das X11-Protokoll funktioniert aber nicht nur lokal.
Es kann auch über ein TCP/IP-Netzwerk genutzt werden.
Auf einem Server kann somit eine X11-Client laufen der von anderen Arbeitsplätzen auf denen dann der X11-Server läuft genutzt werden.\par
Das X11-Protokoll regelt dabei die Kommunikation zwischen Client und Server\cite{xwindow:2023}\cite{x11:2012}.

\section{Repo}
In diesem Repo auf Gitlab befindet sich der Programmierteil der Aufgabe 10. In dieser Aufgabe sollte ein Fenster mit der library für gtk programmiert werden.\par
\href{https://gitlab.thga.de/daniel.krueger/pruefung_sose_2023_aufgabe_10_gui}{\textbf{LINK}}
